%% -------------------------------------------------------------------------------------
%% tcc-poster.tex -- MAIN FILE
%% -------------------------------------------------------------------------------------

% Based from example code found in
% http://www-i6.informatik.rwth-aachen.de/~dreuw/latexbeamerposter.php
%=========================================================================
\documentclass[final,brazil]{beamer}
\usepackage{grffile}
\mode<presentation>{\usetheme{I6dv}}
\usepackage[brazil]{babel}
\usepackage[utf8]{inputenc}
%\usepackage{amsmath,amsthm, amssymb, latexsym}
%\usepackage{times}\usefonttheme{professionalfonts}  % obsolete
%\usefonttheme[onlymath]{serif}
%\boldmath
\usepackage[orientation=landscape,size=a2,scale=1,debug]{beamerposter}
% change list indention level
% \setdefaultleftmargin{3em}{}{}{}{}{}

\usepackage{upquote}
\usepackage{array,booktabs,tabularx}
\usepackage[unicode]{hyperref}

\usepackage{subcaption}         % \begin{subfigure}
\usepackage{wrapfig}            % \begin{wrapfigure}
\usepackage{blindtext}

% centered tabularx columns
\newcolumntype{Z}{>{\centering\arraybackslash}X}
% phantom introduces a vertical space in p formatted table columns??!!
\newcommand{\pphantom}{\textcolor{ta3aluminium}}

\newcommand{\mytilde}{\raise.17ex\hbox{$\scriptstyle\mathtt{\sim}$}}

\newcommand{\blockouterspace}{\vskip1ex}

\newenvironment{innercol}[3][.97\textwidth]{
\begin{columns}
  \hspace{#2}
  \begin{column}{#1}
    \vspace{#3}
    \justifying
}{
  \end{column}
\end{columns}
}

%%%%%%%%%%%%%%%%%%%%%%%%%%%%%%%%%%%%%%%%%%%%%%%%%%%%%%%%%%%%%%%%%%%%%%%%%%%%%%%%%%%%%%
\graphicspath{{figs/}}

\title{\huge Anatomia do BitTorrent: a Ciência da Computação por trás do protocolo}
\author{Paulo Cheadi Haddad Filho --- \,\mytilde paulochf}
\institute[Universidade de São Paulo]
{Orientador: José Coelho de Pina --- \,\mytilde coelho \hspace*{\;\;\qquad}}

%%%%%%%%%%%%%%%%%%%%%%%%%%%%%%%%%%%%%%%%%%%%%%%%%%%%%%%%%%%%%%%%%%%%%%%%%%%%%%%%%%%%%%
\newlength{\columnheight}
\setlength{\columnheight}{105cm}


%%%%%%%%%%%%%%%%%%%%%%%%%%%%%%%%%%%%%%%%%%%%%%%%%%%%%%%%%%%%%%%%%%%%%%%%%%%%%%%%%%%%%%
\begin{document}
\begin{frame}
  \begin{columns}
    % ---------------------------------------------------------%
    % Set up a column
    \begin{column}{.245\textwidth}
      \begin{beamercolorbox}[center,wd=\textwidth]{postercolumn}
        \begin{minipage}[T]{.95\textwidth}  % tweaks the width, makes a new \textwidth
          \parbox[t][\columnheight]{\textwidth}{ % must be some better way to set the the height, width and textwidth simultaneously
            % Since all columns are the same length, it is all nice and tidy.  You have to get the height empirically
            % ---------------------------------------------------------%
            % fill each column with content

            \begin{block}{Objetivo}
              \begin{innercol}{-4ex}{-1.4ex}
                \begin{itemize}
                  \justifying
                  \item \textbf{estudar} o protocolo do BitTorrent
                  \item \textbf{identificar} seu caráter interdisciplinar na Computação
                    pelo estudo de seus componentes internos
                  \item \textbf{ilustrar} a implementação desses componentes utilizando
                    o código fonte do cliente Transmission
                \end{itemize}
              \end{innercol}
            \end{block}

            \blockouterspace

            \begin{block}{Introdução}
              \begin{innercol}{-4ex}{-1.4ex}
                \begin{itemize}
                  \justifying
                  \item \textbf{redes \emph{peer-to-peer} (P2P)}: redes de arquitetura
                descentralizada (sem um servidor central) e distribuída entre vários
                nós da rede \\ (\emph{peers})
                  \item \textbf{Napster} foi a primeira rede P2P, em 1999
                \end{itemize}
              \end{innercol}
            \end{block}

            \blockouterspace

            \begin{block}{História do BitTorrent}
              \begin{innercol}{-4ex}{-1.4ex}
                \begin{itemize}
                  \justifying
                  \item lançado por Bram Cohen em \textbf{2001}

                  \item protocolo P2P \textbf{mais usado no mundo}, que gera
                    $\approx \!\!\: 23 \%$ do tráfego de upload, $\approx \!\! 17 \%$ de
                    download e $\approx \!\! 10 \%$ de todo o tráfego na América Latina
                    \cite{sandvine}

                  \item usado por \textbf{Twitter} \cite{torrentfreak:twitter} e
                    \textbf{Facebook} \cite{torrentfreak:facebook} para distribuir
                    os códigos dos seus sites para seus servidores

                  \item baseado em \textbf{trocas justas} de arquivos e
                    \textbf{comunicação eficiente} entre \emph{peers}

                  \item peers \textbf{sugadores} (\emph{leechers}) e \textbf{semeadores}
                    (\emph{seeders}), pertencentes a um \textbf{enxame} (\emph{swarm}),
                    \textbf{trocam partes} de um torrent entre si

                  \item \textbf{listas de \emph{peers}} mantidas por
                    \textbf{rastreadores} (\emph{trackers}) e, geralmente, pelos
                    próprios \emph{peers}
                \end{itemize}
              \end{innercol}
            \end{block}

            \blockouterspace

            \begin{block}{Transmission}
              \begin{innercol}{-4ex}{-1.4ex}
                \begin{wrapfigure}{r}{.2\linewidth}
                  \vspace{-1ex}
                  \includegraphics[width=\linewidth]{transmission.png}
                \end{wrapfigure}

                \begin{itemize}
                  \justifying
                  \vspace{-3ex}
                  \item \textbf{programa cliente} de \textbf{código aberto} para o
                    protocolo BitTorrent
                  \item escrito nas linguagens \textbf{C} e \textbf{C++}
                  \item \textbf{várias plataformas}: \emph{daemon} (serviço de segundo
                    plano), roteadores, linha de comando e aplicação em janela
                \end{itemize}
              \end{innercol}
            \end{block}

            \blockouterspace

            \begin{block}{BitTorrent: visão do usuário e áreas da Computação}
              \begin{figure}
                \centering
                \includegraphics[width=\textwidth]{funcionamento.png}
              \end{figure}

              \begin{innercol}{-6ex}{-2ex}
                \begin{enumerate}
                  % Enumerate não tá funcionando =P
                  \justifying
                  \item \textbf{1.} busca de conteúdo em sites buscadores de torrent
                  \item \textbf{2.} obtenção do arquivo .torrent desejado
                  \item \textbf{3.} computador do usuário se comunica com rastreadores
                    (\emph{trackers}), que mantêm listas dos \emph{peers} que estão
                    compartilhando os arquivos do torrent obtido
                  \item \textbf{4.} o \emph{tracker} devolve uma lista de \emph{peers}
                    aleatória
                  \item \textbf{5.} computador do usuário inicia comunicação com os
                    \emph{peers} da lista, e começa a receber deles os arquivos
                    pertencentes àquele torrent
                \end{enumerate}
              \end{innercol}
            \end{block}
          }
        \end{minipage}
      \end{beamercolorbox}
    \end{column}
    % ---------------------------------------------------------%
    % end the column

    % ---------------------------------------------------------%
    % Set up a column
    \begin{column}{.49\textwidth}
      \begin{beamercolorbox}[center,wd=\textwidth]{postercolumn}
        \begin{minipage}[T]{.95\textwidth}  % tweaks the width, makes a new \textwidth
          \parbox[t][\columnheight]{\textwidth}{ % must be some better way to set the the height, width and textwidth simultaneously
            % Since all columns are the same length, it is all nice and tidy.  You have to get the height empirically
            % ---------------------------------------------------------%
            % fill each column with content

            \begin{block}{BitTorrent: visão da Internet}
              \begin{innercol}[.99\textwidth]{2ex}{0ex}
                \begin{figure}
                  \centering
                  \includegraphics[width=\textwidth]{universobt.png}
                  %\caption{Exemplo de grafo $G_i$.}
                  %\label{fig:Gi}
                \end{figure}
              \end{innercol}
            \end{block}

            \blockouterspace

            \begin{block}{Trocas das partes de um arquivo torrent em um swarm}
              \begin{innercol}[\textwidth]{-2ex}{0ex}
                \begin{figure}[H]
                  \vspace{-1.75ex}
                  \newlength{\myvsize}
                  \newlength{\myhsize}
                  \setlength{\myvsize}{1.5mm}
                  \setlength{\myhsize}{.235\textwidth}

                  \centering

                  \begin{subfigure}[H]{\myhsize}
                    \fbox{\includegraphics[width=\textwidth]{Torrentcomp_small-0.png}}
                  \end{subfigure}%
                  \quad %add desired spacing between images (~, \quad, \qquad or blank line)
                  \begin{subfigure}[H]{\myhsize}
                    \fbox{\includegraphics[width=\textwidth]{Torrentcomp_small-1.png}}
                  \end{subfigure}%
                  \quad
                  \begin{subfigure}[H]{\myhsize}
                    \fbox{\includegraphics[width=\textwidth]{Torrentcomp_small-2.png}}
                  \end{subfigure}%
                  \quad
                  \begin{subfigure}[H]{\myhsize}
                    \fbox{\includegraphics[width=\textwidth]{Torrentcomp_small-3.png}}
                  \end{subfigure}%

                  \vspace{\myvsize}

                  \begin{subfigure}[H]{\myhsize}
                    \fbox{\includegraphics[width=\textwidth]{Torrentcomp_small-4.png}}
                  \end{subfigure}%
                  \quad
                  \begin{subfigure}[H]{\myhsize}
                    \fbox{\includegraphics[width=\textwidth]{Torrentcomp_small-5.png}}
                  \end{subfigure}%
                  \quad
                  \begin{subfigure}[H]{\myhsize}
                    \fbox{\includegraphics[width=\textwidth]{Torrentcomp_small-6.png}}
                  \end{subfigure}%
                  \quad
                  \begin{subfigure}[H]{\myhsize}
                    \fbox{\includegraphics[width=\textwidth]{Torrentcomp_small-7.png}}
                  \end{subfigure}%

                  \vspace{\myvsize}

                  \begin{subfigure}[H]{\myhsize}
                    \fbox{\includegraphics[width=\textwidth]{Torrentcomp_small-8.png}}
                  \end{subfigure}%
                  \quad
                  \begin{subfigure}[H]{\myhsize}
                    \fbox{\includegraphics[width=\textwidth]{Torrentcomp_small-9.png}}
                  \end{subfigure}%
                \end{figure}
              \end{innercol}
            \end{block}
          }
        \end{minipage}
      \end{beamercolorbox}
    \end{column}
    % ---------------------------------------------------------%
    % end the column

    % ---------------------------------------------------------%
    % Set up a column
    \begin{column}{.24\textwidth}
      \begin{beamercolorbox}[center,wd=\textwidth]{postercolumn}
        \begin{minipage}[T]{.95\textwidth}  % tweaks the width, makes a new \textwidth
          \parbox[t][\columnheight]{\textwidth}{ % must be some better way to set the the height, width and textwidth simultaneously
            % Since all columns are the same length, it is all nice and tidy.  You have to get the height empirically
            % ---------------------------------------------------------%
            % fill each column with content

            \begin{block}{Funcionamento do BitTorrent}
              \begin{innercol}{-6ex}{-1.4ex}
                \begin{enumerate}
                  \justifying
                  \item \textbf{1.} \textbf{busca} de conteúdo em sites buscadores de
                  torrent

                  \item \textbf{2.} \textbf{obtenção} do arquivo .torrent, que é um
                    \textbf{dicionário de dados} sobre os arquivos desejados, onde estão
                    contidas informações como \emph{announces} (endereços de
                    \emph{trackers}), número de partes e lista dos arquivos

                  \item \textbf{3.} \textbf{cliente BitTorrent} usa o \emph{announce}
                    para fazer um pedido de \emph{peers}, usando uma \textbf{requisição HTTP GET} e passando dois identificadores: um seu e o do torrent \\
                    Ex: \url{udp://tracker.publicbt.com:80/announce?info_hash=<hash-do-torrent>&peer_id=<hash-do-cliente>}

                  \item \textbf{4.} o \textbf{\emph{tracker} adiciona} o \emph{peer}
                    requisitante à sua lista e \textbf{devolve} para ele um dicionário
                    contendo: uma \textbf{lista aleatória de outros \emph{peers}} para
                    aquele identificador de torrent, quantidades de \emph{peers},
                    \emph{seeders} e \emph{leechers}, entre outros dados \\
                    Ex: \{`complete': 1, `downloaded': 11, `incomplete': 6, `interval':
                    1732, `min interval': 866, `peers': \{lista-de-hash-ids-de-peers\}\}

                  \item \textbf{5.} o \textbf{cliente}, para entrar no \emph{swarm},
                    \textbf{envia mensagens} para cada um dos \emph{peers} recebidos,
                    até obter uma resposta

                  \begin{enumerate}
                    \item \textbf{5.1} quando \textbf{obtém resposta}, \textbf{adiciona
                      o endereço} desta à sua ``lista de contatos'', que é
                      acessível por outros \emph{peers}, e implementada como uma tabela
                      hash especial, do algoritmo Kademlia. Esse algoritmo faz com que
                      uma busca por \emph{novos peers} seja distribuída entre os que já
                      são conhecidos, formando assim uma grande ``lista telefônica''.
                      Essa estrutura de dados é chamada de \textbf{tabela hash
                      distribuída (DHT)}

                    \item \textbf{5.2} após o primeiro contato, o \textbf{cliente
                      entra} no \emph{swarm} e \textbf{recebe} uma parte aleatória do
                      torrent, se tornando um \emph{leecher}

                    \item \textbf{5.3} como \emph{leecher}, o \textbf{cliente entra} no
                      ``jogo de trocas'' do \emph{swarm}, onde \textbf{só recebe uma
                      parte do torrent se fornecer outra}

                    \item \textbf{5.4} ao \textbf{término} do download, o usuário
                      \textbf{se torna um \emph{seeder}}, passando a unicamente
                      fornecer partes do torrent
                  \end{enumerate}
                \end{enumerate}
              \end{innercol}
            \end{block}

            \blockouterspace

            \begin{block}{ABC no BCC: disciplinas relacionadas}
              \begin{innercol}{-4ex}{-1.4ex}
                \begin{itemize}
                  \justifying
                  \item \textbf{Desenv. de Algoritmos, Estrutura de Dados, Análise de \\
                    Algoritmos}: tabela hash (DHT), listas ligadas, árvores e
                    estruturas compostas

                  \item \textbf{Prog. para Redes}: protocolos de rede HTTP e UDP,
                    segurança SSL, conexões de rede IPv4 e IPv6, roteamento (NAT PMP),
                    troca de mensagens entre \emph{peers}, descoberta de \emph{peers}
                    locais (Multicast)

                  \item \textbf{Intro. à Criptografia}: integridade e privacidade de
                    dados (SHA-1, RC4)

                  \item \textbf{Sistemas Operacionais, Prog. Concorrente, Intro. à Comp.
                    Paralela e Distribuída}: \emph{threads}, DHT, leitura e escrita de
                    partes de arquivo, retomada de download (\emph{download resume})

                  \item \textbf{Lab. de Programação 1--2}: organização do código,
                    Automake, \\ Autoconf, testes de código
                \end{itemize}
              \end{innercol}
            \end{block}

            \blockouterspace

            \begin{block}{Referências}
              \begin{innercol}{-2.5ex}{0ex}
                \small{
                  \begin{thebibliography}{99}
                    \bibitem{sandvine}
                      Sandvine Inc.
                      \textbf{Global Internet Phenomena Report --- 1H 2013}, 2013. \\
                      \url{http://macaubas.com/wp-content/uploads/2013/05/Sandvine_Global_Internet_Phenomena_Report_1H_2013.pdf}

                    \bibitem{torrentfreak:twitter}
                      Ernesto.
                      \textbf{BitTorrent Makes Twitter’s Server Deployment 75x Faster},
                      16 de julho de 2013. \emph{TorrentFreak}\\
                      \url{http://torrentfreak.com/bittorrent-makes-twitters-server-deployment-75-faster-100716/}

                    \bibitem{torrentfreak:facebook}
                      Ernesto.
                      \textbf{Facebook Uses BitTorrent, and They Love It},
                      25 de junho de 2013. \emph{TorrentFreak}\\
                      \url{http://torrentfreak.com/facebook-uses-bittorrent-and-they-love-it-100625/}

                    \bibitem{wiki:timeline}
                      Wikipedia.
                      \textbf{Timeline of file sharing} --- \emph{Wikipedia{,} The
                      Free Encyclopedia}, 2013.
                      \url{http://en.wikipedia.org/wiki/Timeline_of_file_sharing}

                    \bibitem{swarming}
                      Matheus B. Lehmann, Rodrigo B. Mansilha, Marinho P. Barcellos e
                      Flávio Roberto Santos.
                      \textbf{``Swarming: como BitTorrent revolucionou a Internet''}.
                      Em \emph{Atualizações em Informática}. PUC-Rio. Vol. 1. Rio de
                      Janeiro, 2011. Cap. 6, pp. 209–258
                  \end{thebibliography}
                }
              \end{innercol}
            \end{block}
          }
        \end{minipage}
      \end{beamercolorbox}
    \end{column}
    % ---------------------------------------------------------%
    % end the column

  \end{columns}
  \vskip1ex
  %\tiny\hfill\textcolor{ta2gray}{Created with \LaTeX \texttt{beamerposter}  \url{http://www-i6.informatik.rwth-aachen.de/~dreuw/latexbeamerposter.php}}
  \tiny\hfill{Created with \LaTeX \texttt{beamerposter}  \url{http://www-i6.informatik.rwth-aachen.de/~dreuw/latexbeamerposter.php} \hskip1em}
\end{frame}
\end{document}

%%%%%%%%%%%%%%%%%%%%%%%%%%%%%%%%%%%%%%%%%%%%%%%%%%%%%%%%%%%%%%%%%%%%%%%%%%%%%%%%%%%%%%%%%%%%%%%%%%%%
%%% Local Variables:
%%% mode: latex
%%% TeX-PDF-mode: t
%%% End: