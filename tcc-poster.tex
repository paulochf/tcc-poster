%% -------------------------------------------------------------------------------------
%% tcc-poster.tex -- MAIN FILE
%% -------------------------------------------------------------------------------------

% Based from example code found in
% http://www-i6.informatik.rwth-aachen.de/~dreuw/latexbeamerposter.php
%=========================================================================
\documentclass[final,brazil]{beamer}
\usepackage{grffile}
\mode<presentation>{\usetheme{I6dv}}
\usepackage[brazil]{babel}
\usepackage[utf8]{inputenc}
%\usepackage{amsmath,amsthm, amssymb, latexsym}
%\usepackage{times}\usefonttheme{professionalfonts}  % obsolete
%\usefonttheme[onlymath]{serif}
%\boldmath
\usepackage[orientation=landscape,size=a2,scale=1,debug]{beamerposter}
% change list indention level
% \setdefaultleftmargin{3em}{}{}{}{}{}

\usepackage[backend=biber,bibencoding=utf8,style=numeric-comp]{biblatex}
\bibliography{bibliografia.bib}

\usepackage{array,booktabs,tabularx}
\usepackage[unicode]{hyperref}

\usepackage{subcaption}         % \begin{subfigure}
\usepackage{wrapfig}            % \begin{wrapfigure}
\usepackage{blindtext}

% centered tabularx columns
\newcolumntype{Z}{>{\centering\arraybackslash}X}
% phantom introduces a vertical space in p formatted table columns??!!
\newcommand{\pphantom}{\textcolor{ta3aluminium}}

\newcommand{\mytilde}{\raise.17ex\hbox{$\scriptstyle\mathtt{\sim}$}}

\newcommand{\blockouterspace}{\vskip1ex}

\newenvironment{innercol}{
\begin{columns}
  \begin{column}{.93\textwidth}
    \justifying
}{
  \end{column}
\end{columns}
}

%%%%%%%%%%%%%%%%%%%%%%%%%%%%%%%%%%%%%%%%%%%%%%%%%%%%%%%%%%%%%%%%%%%%%%%%%%%%%%%%%%%%%%
\graphicspath{{figs/}{figs/teste/}}

\title{\huge Anatomia do BitTorrent: a Ciência da Computação por trás do protocolo}
\author{Paulo Cheadi Haddad Filho --- \,\mytilde paulochf}
\institute[Universidade de São Paulo]
{Orientador: José Coelho de Pina --- \,\mytilde coelho \hspace*{\;\;\qquad}}

%%%%%%%%%%%%%%%%%%%%%%%%%%%%%%%%%%%%%%%%%%%%%%%%%%%%%%%%%%%%%%%%%%%%%%%%%%%%%%%%%%%%%%
\newlength{\columnheight}
\setlength{\columnheight}{105cm}


%%%%%%%%%%%%%%%%%%%%%%%%%%%%%%%%%%%%%%%%%%%%%%%%%%%%%%%%%%%%%%%%%%%%%%%%%%%%%%%%%%%%%%
\begin{document}
\begin{frame}
  \begin{columns}
    % ---------------------------------------------------------%
    % Set up a column
    \begin{column}{.245\textwidth}
      \begin{beamercolorbox}[center,wd=\textwidth]{postercolumn}
        \begin{minipage}[T]{.95\textwidth}  % tweaks the width, makes a new \textwidth
          \parbox[t][\columnheight]{\textwidth}{ % must be some better way to set the the height, width and textwidth simultaneously
            % Since all columns are the same length, it is all nice and tidy.  You have to get the height empirically
            % ---------------------------------------------------------%
            % fill each column with content

            \begin{block}{Introdução}
  \begin{innercol}
    As redes \emph{peer-to-peer} (P2P) são redes de arquitetura
    descentralizada (sem um servidor central) e distribuída entre vários
    nós da rede (os \emph{peers}), os quais sugadores (\emph{leechers})
    e semeadores (\emph{seeders}) trocam dados entre si. A primeira rede
    P2P surgiu com o Napster, em 1999.
  \end{innercol}

              %\begin{innercol}
                \begin{itemize}
                  \item \textbf{redes \emph{peer-to-peer} (P2P)}: redes de arquitetura
                descentralizada (sem um servidor central) e distribuída entre vários
                nós da rede (\emph{peers})
                  \item \textbf{Napster} foi a primeira rede P2P, em 1999

                \end{itemize}
              %\end{innercol}
            \end{block}

            \blockouterspace

            \begin{block}{História do BitTorrent}
              \begin{innercol}
                Em meados de 2001, Bram Cohen lançou o BitTorrent, rede P2P que se
                baseia em trocas justas de arquivos entre usuários, e que se comunicam
                de forma eficiente a fim de localizar mais fontes dos arquivos
                desejados. Atualmente, após melhorias, é o meio de compartilhamento de
                arquivos mais usado no mundo, sendo responsável por grande tráfego de
                dados na Internet.
              \end{innercol}

              \begin{itemize}
                \item lançado por Bram Cohen em \textbf{2001}
                \item protocolo P2P mais usado no mundo, que gera $\approx23\%$ do
                tráfego de upload, $\approx17\%$ de download e $\approx10\%$ na América
                Latina \cite{}
                \item uso não-convencional: Twitter e Facebook distribuem o código pelos
                  seus servidores \cite{}
                \item baseado em trocas justas de arquivos e comunicação eficiente entre
                  \emph{peers}
                \item peers \textbf{sugadores} (\emph{leechers}) e \textbf{semeadores} (\emph{seeders}) trocam dados entre si
                \item listas de \emph{peers} mantidas por rastreadores (\emph{trackers})
                e geralmente pelos próprios \emph{peers}
              \end{itemize}
            \end{block}

            \blockouterspace

            \begin{block}{BitTorrent: visão do usuário e áreas da Computação}
              \begin{figure}
                \centering
                \includegraphics[width=\textwidth]{funcionamento.png}
                %\caption{Exemplo de grafo $G_i$.}
                %\label{fig:Gi}
              \end{figure}

              %\begin{innercol}
                % Enumerate não tá funcionando =P
                \begin{enumerate}
                  \item 1. busca de conteúdo em sites buscadores de torrent
                  \item 2. obtenção do arquivo .torrent desejado
                  \item 3. computador do usuário se comunica com rastreadores \\
                  (\emph{trackers}), que mantêm listas dos \emph{peers} que estão \\
                  compartilhando os arquivos do torrent obtido
                  \item 4. o \emph{tracker} devolve uma lista de \emph{peers} aleatória
                  \item 5. computador do usuário inicia comunicação com os \emph{peers}
                  \\ da lista, e começa a receber deles os arquivos pertencentes \\
                  àquele torrent
                \end{enumerate}
              %\end{innercol}
            \end{block}

            \blockouterspace

            \begin{block}{Transmission}
              \begin{innercol}
                \begin{wrapfigure}{l}{.2\linewidth}
                  \vspace{-1ex}
                  \hspace{-1ex}
                  \includegraphics[width=\linewidth]{transmission.png}
                \end{wrapfigure}
                O Transmission é um programa cliente do protoloco BitTorrent de código
                aberto escrito em linguagem C e C++, com versões específicas desde
                para linha de comando, \emph{daemon} (serviço em segundo plano),
                roteadores e a comumente utilizada aplicação em janela para Linux e
                OS X.
              \end{innercol}
            \end{block}
          }
        \end{minipage}
      \end{beamercolorbox}
    \end{column}
    % ---------------------------------------------------------%
    % end the column

    % ---------------------------------------------------------%
    % Set up a column
    \begin{column}{.49\textwidth}
      \begin{beamercolorbox}[center,wd=\textwidth]{postercolumn}
        \begin{minipage}[T]{.95\textwidth}  % tweaks the width, makes a new \textwidth
          \parbox[t][\columnheight]{\textwidth}{ % must be some better way to set the the height, width and textwidth simultaneously
            % Since all columns are the same length, it is all nice and tidy.  You have to get the height empirically
            % ---------------------------------------------------------%
            % fill each column with content

            \begin{block}{BitTorrent: visão da Internet}
              \begin{figure}
                \centering
                \includegraphics[width=\textwidth]{universobt.png}
                %\caption{Exemplo de grafo $G_i$.}
                %\label{fig:Gi}
              \end{figure}
            \end{block}

            \blockouterspace

            \begin{block}{Funcionamento do BitTorrent}
              \begin{columns}[t,\textwidth]
                \begin{column}{.46\textwidth}
                  \justifying
                  \blindtext
                  \blindtext
                \end{column}

                \begin{column}{.46\textwidth}
                  \justifying
                  \blindtext
                  \blindtext
                \end{column}
              \end{columns}
            \end{block}
          }
        \end{minipage}
      \end{beamercolorbox}
    \end{column}
    % ---------------------------------------------------------%
    % end the column

    % ---------------------------------------------------------%
    % Set up a column
    \begin{column}{.24\textwidth}
      \begin{beamercolorbox}[center,wd=\textwidth]{postercolumn}
        \begin{minipage}[T]{.95\textwidth}  % tweaks the width, makes a new \textwidth
          \parbox[t][\columnheight]{\textwidth}{ % must be some better way to set the the height, width and textwidth simultaneously
            % Since all columns are the same length, it is all nice and tidy.  You have to get the height empirically
            % ---------------------------------------------------------%
            % fill each column with content

            \begin{block}{Anatomia}
              \begin{innercol}
                \blindenumerate[10]
              \end{innercol}
            \end{block}

            \blockouterspace

            \begin{block}{Simulação}
              \begin{innercol}
                \begin{figure}[H]
                  \newlength{\myvsize}
                  \newlength{\myhsize}
                  \setlength{\myvsize}{4mm}
                  \setlength{\myhsize}{.3\textwidth}

                  \centering

                  \begin{subfigure}[H]{\myhsize}
                    \fbox{\includegraphics[width=\textwidth]{Torrentcomp_small-0.png}}
                    \caption{}
                    \label{fig:torrent-repr-0}
                  \end{subfigure}%
                  \quad %add desired spacing between images (~, \quad, \qquad or blank line)
                  \begin{subfigure}[H]{\myhsize}
                    \fbox{\includegraphics[width=\textwidth]{Torrentcomp_small-1.png}}
                    \caption{}
                    \label{fig:torrent-repr-1}
                  \end{subfigure}%
                  \quad
                  \begin{subfigure}[H]{\myhsize}
                    \fbox{\includegraphics[width=\textwidth]{Torrentcomp_small-2.png}}
                    \caption{}
                    \label{fig:torrent-repr-2}
                  \end{subfigure}

                  \vspace{\myvsize}

                  \begin{subfigure}[H]{\myhsize}
                    \fbox{\includegraphics[width=\textwidth]{Torrentcomp_small-3.png}}
                    \caption{}
                    \label{fig:torrent-repr-3}
                  \end{subfigure}%
                  \quad
                  \begin{subfigure}[H]{\myhsize}
                    \fbox{\includegraphics[width=\textwidth]{Torrentcomp_small-4.png}}
                    \caption{}
                    \label{fig:torrent-repr-4}
                  \end{subfigure}%
                  \quad
                  \begin{subfigure}[H]{\myhsize}
                  \fbox{\includegraphics[width=\textwidth]{Torrentcomp_small-5.png}}
                  \caption{}
                  \label{fig:torrent-repr-5}
                  \end{subfigure}

                  \vspace{\myvsize}

                  \begin{subfigure}[H]{\myhsize}
                    \fbox{\includegraphics[width=\textwidth]{Torrentcomp_small-6.png}}
                    \caption{}
                    \label{fig:torrent-repr-6}
                  \end{subfigure}%
                  \quad
                  \begin{subfigure}[H]{\myhsize}
                    \fbox{\includegraphics[width=\textwidth]{Torrentcomp_small-7.png}}
                    \caption{}
                    \label{fig:torrent-repr-7}
                  \end{subfigure}%
                  \quad
                  \begin{subfigure}[H]{\myhsize}
                    \fbox{\includegraphics[width=\textwidth]{Torrentcomp_small-8.png}}
                    \caption{}
                    \label{fig:torrent-repr-8}
                  \end{subfigure}

                  \vspace{\myvsize}

                  \begin{subfigure}[H]{\myhsize}
                    \fbox{\includegraphics[width=\textwidth]{Torrentcomp_small-9.png}}
                    \caption{}
                    \label{fig:torrent-repr-9}
                  \end{subfigure}
                \end{figure}
              \end{innercol}
            \end{block}

            \blockouterspace

            \begin{block}{Referências}
              \begin{innercol}
                \scriptsize{
                  \begin{thebibliography}{99}
                    \bibitem{WS}
                      D.P. Williamson, and D.B. Shmoys,
                    ``\textbf{The Design of Approximation Algorithms},'' \emph{Cambridge
                      University Press}, 2011.

                    \bibitem{HS}
                    D.Hochbaum, and D.B.Shmoys,
                    ``\textbf{A unified approach to approximation algorithms for bottleneck
                    problems},'' in \emph{Journal of the ACM}, 1986, volume 33, pp. 533-550.

                    \bibitem{CLRS}
                    T.H.Cormen, C.E.Leiserson, R.L.Rivest, and C. Stein,
                    ``\textbf{Introduction to algorithms}'', \emph{The MIT Press}, ed. 3, 2009.

                  \end{thebibliography}
                }
              \end{innercol}
            \end{block}
          }
        \end{minipage}
      \end{beamercolorbox}
    \end{column}
    % ---------------------------------------------------------%
    % end the column

  \end{columns}
  \vskip1ex
  %\tiny\hfill\textcolor{ta2gray}{Created with \LaTeX \texttt{beamerposter}  \url{http://www-i6.informatik.rwth-aachen.de/~dreuw/latexbeamerposter.php}}
  \tiny\hfill{Created with \LaTeX \texttt{beamerposter}  \url{http://www-i6.informatik.rwth-aachen.de/~dreuw/latexbeamerposter.php} \hskip1em}
\end{frame}
\end{document}

%%%%%%%%%%%%%%%%%%%%%%%%%%%%%%%%%%%%%%%%%%%%%%%%%%%%%%%%%%%%%%%%%%%%%%%%%%%%%%%%%%%%%%%%%%%%%%%%%%%%
%%% Local Variables:
%%% mode: latex
%%% TeX-PDF-mode: t
%%% End: